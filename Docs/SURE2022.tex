\documentclass[11pt,reqno]{amsart}
\usepackage{mathptmx}
\usepackage{mathtools}
\usepackage{mathrsfs}
\usepackage[all]{xy}
\usepackage{stmaryrd}
\usepackage{fancyhdr}

\usepackage{pgf, tikz}
\usetikzlibrary{arrows, automata}

\usepackage{tikz-cd}
\usepackage{cite}
\usepackage{amsmath,amsfonts}
\usepackage{graphicx}
\usepackage{wrapfig}
\usepackage{array}
\usepackage{amsmath,amsthm,amssymb,hyperref}
\usepackage{exercise}

\usepackage{enumerate}
\usepackage{tikz}
\pagestyle{plain}
\usepackage[left=1.2in, right=1.2in, top=1in, bottom=1in]{geometry}
\usepackage{etoolbox}
\patchcmd{\section}{\scshape}{\bfseries}{}{}
\makeatletter
\renewcommand{\@secnumfont}{\bfseries}
\makeatother
\xyoption{all}
\thispagestyle{empty}

\DeclareMathOperator{\Hom}{Hom}
\DeclareMathOperator{\Spec}{Spec}
\DeclareMathOperator{\Spv}{Spv}
\DeclareMathOperator{\Frac}{Frac}
\DeclareMathOperator{\Img}{Img}
\DeclareMathOperator{\Ker}{Ker}
\DeclareMathOperator{\Pic}{Pic}
\DeclareMathOperator{\Div}{Div}
\DeclareMathOperator{\dv}{Div}
\DeclareMathOperator{\Deg}{deg}
\DeclareMathOperator{\Aut}{Aut}
\DeclareMathOperator{\CSpec}{Cspec}
\newcommand{\angles}[1]{\langle #1 \rangle}
\newcommand{\Jac}{\textrm{Jac}}{}
\newcommand{\R}{\mathbb{R}}  
\newcommand{\Z}{\mathbb{Z}}
\newcommand{\N}{\mathbb{N}}
\newcommand{\Q}{\mathbb{Q}}
\newcommand{\I}{\mathbf{I}}

\input xy
\xyoption{all}
\thispagestyle{empty}


%\usepackage{secdot} 

\theoremstyle{definition}
\newtheorem{mydef}{\textbf{Definition}}[section]
\newtheorem{myeg}[mydef]{\textbf{Example}}
\newtheorem{conj}[mydef]{\textbf{Conjecture}}
\newtheorem*{noconj}{\textbf{Conjecture}}
\newtheorem{observ}[mydef]{\textbf{Observation}}
\newtheorem{question}[mydef]{\textbf{Question}}
\newtheorem{rmk}[mydef]{\textbf{Remark}}
\newtheorem*{que}{\textbf{Question}}
\newtheorem*{goal}{\textbf{Goal}}

\theoremstyle{plain}
\newtheorem{mythm}[mydef]{\textbf{Theorem}}
\newtheorem*{nothm}{\textbf{Theorem}}
\newtheorem*{nomainthm}{\textbf{Main Theorem}}
\newtheorem*{nothma}{\textbf{Theorem A}}
\newtheorem*{nothmb}{\textbf{Theorem B}}
\newtheorem*{nothmc}{\textbf{Theorem C}}
\newtheorem*{nothmd}{\textbf{Theorem D}}
\newtheorem*{nothme}{\textbf{Theorem E}}
\newtheorem*{nothmf}{\textbf{Theorem F}}
\newtheorem*{nothmg}{\textbf{Theorem G}}
\newtheorem{mytheorem}[mydef]{\textbf{Theorem}}
\newtheorem{lem}[mydef]{\textbf{Lemma}}
\newtheorem{pro}[mydef]{\textbf{Proposition}}
\newtheorem{claim}[mydef]{\textbf{Claim}}
\newtheorem{cor}[mydef]{\textbf{Corollary}}
\newtheorem{con}[mydef]{\textbf{Construction}}


\patchcmd{\abstract}{\scshape\abstractname}{\normalsize{\textbf{\abstractname}}}{}{}
\begin{document}


\title{Summer project}
%

\author{Jaiung Jun}
\address{Department of Mathematics, State University of New York at New Paltz, NY 12561, USA}
\email{junj@newpaltz.edu}

\author{Matthew Pisano}
\address{State University of New York at Binghamton, NY 13902, USA}
\email{pisanom1@newpaltz.edu}


%

%
\makeatletter
\@namedef{subjclassname@2020}{%
	\textup{2020} Mathematics Subject Classification}
\makeatother

\subjclass[2020]{05C50, 05C76}
\keywords{Jacobian of a graph, sandpile group, critical group, chip-firing game, gluing graphs, cycle graph, Tutte polynomial, Tutte's rotor construction}
%\thanks{}

\begin{abstract}

\end{abstract}

\maketitle


\section{Introduction}


We will focus on Research Project 11 in \cite{glass2020chip}. The following are more specific directions that we plan to pursue.


\begin{goal}[8/1/2022]$ $
	\begin{enumerate}
\item 
Prove/disprove: for an oriented graph $G$, one always has $\Pic(G)=\mathbb{Z} \times \Jac(G)$, i.e., as a finitely generated abelian group, the rank of $\Pic(G)$ is $1$.
		\item 
Prove/disprove: for $C_n$, and $0 \leq m \leq n$, one can always find an orientation of $C_n$ so that $\Jac(C_n)=\mathbb{Z}_m$ (with the orientation). 
\item 
Prove/disprove: for an oriented graph $G$, if $v_0 \in V(G)$ is a sink (or a source) and $G'$ is the graph obtained by reserving the direction for all arrows adjacent to $v_0$ from $G$, then $\Jac(G)=\Jac(G')$. (Note: we believe that this should be true for at least some classes of graphs such as cyclic graphs.)
\item 
Prove/disprove: for an oriented planar graph $G$ and its planar dual (should be defined) $\hat{G}$, one has $\Jac(G)=\Jac(\hat{G})$. 
\item 
Prove/disprove: for oriented graphs $G_1,G_2$, let $G$ be the graph obtained by gluing $G_1$ and $G_2$ along one vertex. Then $\Jac(G)=\Jac(G_1) \times \Jac(G_2)$. 
\item 

	\end{enumerate}
\end{goal}


\section{Preliminaries}

\section{Propositions}





\bibliography{Jacobian}\bibliographystyle{alpha}


\end{document}
