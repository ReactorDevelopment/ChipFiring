\documentclass[11pt,reqno]{amsart}
\usepackage{mathptmx}
\usepackage{mathtools}
\usepackage{mathrsfs}
\usepackage[all]{xy}
\usepackage{stmaryrd}
\usepackage{fancyhdr}

\usepackage{pgf, tikz}
\usetikzlibrary{arrows, automata}

\usepackage{tikz-cd}
\usepackage{cite}
\usepackage{amsmath,amsfonts}
\usepackage{graphicx}
\usepackage{wrapfig}
\usepackage{array}
\usepackage{amsmath,amsthm,amssymb,hyperref}
\usepackage{exercise}

\usepackage{enumerate}
\usepackage{tikz}
\pagestyle{plain}
\usepackage[left=1.2in, right=1.2in, top=1in, bottom=1in]{geometry}
\usepackage{etoolbox}
\patchcmd{\section}{\scshape}{\bfseries}{}{}
\makeatletter
\renewcommand{\@secnumfont}{\bfseries}
\makeatother
\xyoption{all}
\thispagestyle{empty}

\DeclareMathOperator{\Hom}{Hom}
\DeclareMathOperator{\Spec}{Spec}
\DeclareMathOperator{\Spv}{Spv}
\DeclareMathOperator{\Frac}{Frac}
\DeclareMathOperator{\Img}{Img}
\DeclareMathOperator{\Ker}{Ker}
\DeclareMathOperator{\Pic}{Pic}
\DeclareMathOperator{\Div}{Div}
\DeclareMathOperator{\dv}{Div}
\DeclareMathOperator{\Deg}{deg}
\DeclareMathOperator{\Aut}{Aut}
\DeclareMathOperator{\CSpec}{Cspec}
\newcommand{\angles}[1]{\langle #1 \rangle}
\newcommand{\Jac}{\textrm{Jac}}{}
\newcommand{\R}{\mathbb{R}}  
\newcommand{\Z}{\mathbb{Z}}
\newcommand{\N}{\mathbb{N}}
\newcommand{\Q}{\mathbb{Q}}
\newcommand{\I}{\mathbf{I}}

\input xy
\xyoption{all}
\thispagestyle{empty}


%\usepackage{secdot} 

\theoremstyle{definition}
\newtheorem{mydef}{\textbf{Definition}}[section]
\newtheorem{myeg}[mydef]{\textbf{Example}}
\newtheorem{conj}[mydef]{\textbf{Conjecture}}
\newtheorem*{noconj}{\textbf{Conjecture}}
\newtheorem{observ}[mydef]{\textbf{Observation}}
\newtheorem{question}[mydef]{\textbf{Question}}
\newtheorem{rmk}[mydef]{\textbf{Remark}}
\newtheorem*{que}{\textbf{Question}}
\newtheorem*{goal}{\textbf{Goal}}

\theoremstyle{plain}
\newtheorem{mythm}[mydef]{\textbf{Theorem}}
\newtheorem*{nothm}{\textbf{Theorem}}
\newtheorem*{nomainthm}{\textbf{Main Theorem}}
\newtheorem*{nothma}{\textbf{Theorem A}}
\newtheorem*{nothmb}{\textbf{Theorem B}}
\newtheorem*{nothmc}{\textbf{Theorem C}}
\newtheorem*{nothmd}{\textbf{Theorem D}}
\newtheorem*{nothme}{\textbf{Theorem E}}
\newtheorem*{nothmf}{\textbf{Theorem F}}
\newtheorem*{nothmg}{\textbf{Theorem G}}
\newtheorem{mytheorem}[mydef]{\textbf{Theorem}}
\newtheorem{lem}[mydef]{\textbf{Lemma}}
\newtheorem{pro}[mydef]{\textbf{Proposition}}
\newtheorem{claim}[mydef]{\textbf{Claim}}
\newtheorem{cor}[mydef]{\textbf{Corollary}}
\newtheorem{con}[mydef]{\textbf{Construction}}


\patchcmd{\abstract}{\scshape\abstractname}{\normalsize{\textbf{\abstractname}}}{}{}
\begin{document}


\title{Summer project}
%

\author{Jaiung Jun}
\address{Department of Mathematics, State University of New York at New Paltz, NY 12561, USA}
\email{junj@newpaltz.edu}

\author{Matthew Pisano}
\address{State University of New York at Binghamton, NY 13902, USA}
\email{pisanom1@newpaltz.edu}


%

%
\makeatletter
\@namedef{subjclassname@2020}{%
	\textup{2020} Mathematics Subject Classification}
\makeatother

\subjclass[2020]{05C50, 05C76}
\keywords{Jacobian of a graph, sandpile group, critical group, chip-firing game, gluing graphs, cycle graph, Tutte polynomial, Tutte's rotor construction}
%\thanks{}

\begin{abstract}

\end{abstract}

\maketitle


\section{Introduction}

	We will focus on Research Project 11 in~\cite{glass2020chip}.
	The following are more specific directions that we plan to pursue.

	\begin{goal}[8/1/2022]$ $
		\begin{enumerate}
			\item
			Prove/disprove: for an oriented graph $G$, one always has $\Pic(G)=\mathbb{Z} \times \Jac(G)$,
			i.e., as a finitely generated abelian group, the rank of $\Pic(G)$ is $1$.
			\item
			Prove/disprove: for $C_n$, and $0 \leq m \leq n$, one can always find an orientation
			of $C_n$ so that $\Jac(C_n)=\mathbb{Z}_m$ (with the orientation).
			\item
			Prove/disprove: for an oriented graph $G$, if $v_0 \in V(G)$ is a sink (or a source)
			and $G'$ is the graph obtained by reserving the direction for all arrows adjacent
			to $v_0$ from $G$, then $\Jac(G)=\Jac(G')$. (Note: we believe that this should be true
			for at least some classes of graphs such as cyclic graphs.)
			\item
			Prove/disprove: for an oriented planar graph $G$ and its planar dual (should be defined)
			$\hat{G}$, one has $\Jac(G)=\Jac(\hat{G})$.
			\item
			Prove/disprove: for oriented graphs $G_1,G_2$, let $G$ be the graph obtained by
			gluing $G_1$ and $G_2$ along one vertex. Then $\Jac(G)=\Jac(G_1) \times \Jac(G_2)$.
		\end{enumerate}
	\end{goal}

\section{Preliminaries}
	\subsection{Chip Firing}
		The game at the heart of this paper is the Chip-Firing game. When a game is started, each vertex on
		a graph is assigned a certain number of chips.  During play, chips can be lent or borrowed at each
		node where one or more chips are either sent or received along each outgoing edge equally.  In the
		case of a directed graph, vertices can only interact with another along an outgoing or
		bidirectional edge.  The game is won once every vertex has a positive number of chips (i.e., this
		vertex is not in debt).

	\subsection{Divisors and Equivalence Relations}
		In the study of this game a \textbf{Divisor} of a graph ($\Div(G)$) is an integer vector $v\in\mathbb{Z}^n$
		where \textit{n} is the number of vertices in the graph.  The $i^{th}$ of element of the vector \textit{v}
		is the number of chips on the $i^{th}$ vertex of the graph.  Two divisors have an \textbf{Equivalence Relation}
		($\sim$) if one divisor can be gotten from the other by a finite series of lending or borrowing moves
		$D_1 \sim D_2 \xleftrightarrow{} (D_1 \xleftrightarrow{\text{moves}} D_2)$.  An \textbf{Equivalence Class} \textit{[D]}
		is the set of all divisors that are equivalent to each other, $[D] = \{D_i~|~D_i \sim D\}$.

	\subsection{The Picard Group and The Jacobian}
		The \textbf{Picard Group} of a graph $\Pic(G)$ is the set of all equivalence classes that the divisors of that graph can be a
		part of. The \textbf{Jacobian} of a graph  $\Jac(G)$ is a subset of $\Pic(G)$ such that every divisor in each
		equivalency class has a degree of $0$ where the degree of a divisor $\Deg(D)$ is the sum of each of the divisor's elements.
		If a divisor is in one of the Jacobian's classes, it can be made winning after a finite series of moves.

\section{Propositions}
	\subsection{Calculating Sinks and Sources}
		Vertices on a directed graph can be classified in three ways, as a sink, source, or neither depending on the direction
		of the edges connected to it.  A sink is a vertex where all edges are directed into that vertex, a source is a vertex where
		all edges are directed away from that vertex and a vertex is neither when it has a mixture of the two.  The number of sinks
		and sources can be calculated inductively by reconstructing the original graph from a single vertex.

		Beginning with any two vertices that are adjacent on the original graph, observe their connecting edge.
		If the edge is directed, the count of sinks and sources begins at 1 for both.  If the edge
		is undirected (bidirectional), the count begins at zero. Next, pick a vertex adjacent to one of the original
		edges and observe its connection.  From this point on, when a new edge is observed the number of sinks and
		sources changes on a series of rules.  For these rules, let the vertex being added to be $V_1$ and
		the vertex being added be $V_2$.
		\begin{enumerate}
			\item If $V_1$ is neither a sink nor a source.
			\begin{enumerate}
				\item Adding an edge directed towards $V_2$ adds a sink.
				\item Adding an edge directed away from $V_2$ adds a source.
				\item Adding a bidirectional edge between $V_1$ and $V_2$ adds nothing.
			\end{enumerate}
			\item If $V_1$ is a source.
			\begin{enumerate}
				\item Adding an edge directed towards $V_2$ adds a sink.
				\item Adding an edge directed away from $V_2$ adds nothing (Here the number of sources stays \label{itm:noChange}
					the same as $V_1$ is no longer a source but $V_2$ now is).
				\item Adding a bidirectional edge between $V_1$ and $V_2$ removes a source.
			\end{enumerate}
			\item If $V_1$ is a sink.
			\begin{enumerate}
				\item Adding an edge directed towards $V_2$ adds nothing (See~\ref{itm:noChange}).
				\item Adding an edge directed away from $V_2$ adds a source.
				\item Adding a bidirectional edge between $V_1$ and $V_2$ removes a sink.
			\end{enumerate}
		\end{enumerate}
		When accounting for a vertex that has multiple connections, apply these rules for every $V_{1i}$ that $V_2$
		is connected to, taking into account all the edges of $V_2$ when evaluating its class.  After every vertex has
		been accounted for, the number of sinks and sources will have been calculated in polynomial time between $O(n)$
		at the average case and $O(n^2)$ for the case of a connected graph. \label{subsec:sinkssourceprop}
	\subsection{Using Sinks and Sources to Calculate $Rank(\Pic(G))$}
		$\Pic(G)$ is often in the form $\mathbb{Z}_1 \times \dots \times \mathbb{Z}_n \times \mathbb{Z}^m$ where $m$ is
		the rank of the Picard group.  When calculating $\Pic(G)$, its $Rank()$ always the number of sinks for any
		arbitrary graph.  $Rank(\Pic(G))$ is represented in the smith normal form of the laplacian as the number of
		all zero rows.  For all $Rank(\Pic(G)) \geq 2$, the number of all zero rows in the smith normal form matches
		the number of all zero rows in the laplacian matrix of $G$ or the number of sink vertices in $G$.  The rank of a
		graph's Picard group can be found by applying~\ref{subsec:sinkssourceprop} to keep track of the sinks while
		recreating the original graph.
	\subsection{Directed Cyclic Jacobians}
		Proposition: For $C_n$, and $0 \leq m \leq n$, one can always find an orientation of $C_n$ so
		that $\Jac(C_n)=\mathbb{Z}_m$ (with the orientation).
		Proof: the proof for this proposition was proven incompletely up to $C_{17}$ using a brute force method of checking
		all oriented configurations of directed edges until all $\{\Jac(C_n)=\mathbb{Z}_m~|~0 \leq m \leq n\}$ have been found.
		The first $C_3$ $C_{17}$ were proven in $O(3^n)$ time.  The algorithm used did not have to check all permutations, however,
		only a very cmall portion.  This resulted in each portion being $\frac{1}{3}$ of the portion of the previous graph.
		Due to this, calculations for the first 17 graph was calculated in closer to $O(x^n)$ polynomial time.

\bibliography{Jacobian}
\bibliographystyle{alpha}

\end{document}
